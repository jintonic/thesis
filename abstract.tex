\section*{Abstract}
The results from neutrino oscillation experiments indicate that at least two neutrinos have mass. However, what the absolute mass scale of neutrinos is and whether neutrino and anti-neutrino are their own anti-particles remain unsolved. Neutrinoless double beta decay experiments can help to improve our understanding of both problems and are the most practical method known to tackle the second question. 

The GERmanium Detector Array (GERDA) experiment searching for the neutrinoless double beta decay of $^{76}$Ge is currently under construction in Hall A of the INFN Gran Sasso National Laboratory (LNGS), Italy. In order to achieve an extremely low background level, segmented germanium detectors will be operated directly in liquid argon which serves as cooling and shielding material simultaneously. 

Several test cryostats were built to operate segmented germanium detectors in vacuum and cryogenic liquid at the Max-Planck-Institut f\"ur Physik in M\"unchen, Germany. The performance and the background discrimination power of segmented germanium detectors were studied in detail. It was proved for the first time that the segmented germanium detector can be operated directly in cryogenic liquid stably for a long period, and that the segmentation scheme employed does well in the identification of photon and neutron induced background.

A comprehensive C++ simulation framework, MaGe (Majorana-Gerda), is jointly developed by the Majorana and GERDA collaborations. It is based on Geant4, but tailored to be especially suitable for the simulation of the response of ultra-low radioactive background radiation detectors to ionizing radiation. The predictions of the simulations were verified to hold to 5\%.

Pulse shape analysis is a complementary method to segmentation for further identifying background events. Its identification efficiency needs to be estimated using reliable pulse shape simulations. A full-functional pulse shape simulation package was developed within the MaGe framework. The simulation was verified using data taken from the first segmented prototype detector for GERDA. The understanding of the properties of the segmented germanium detectors was improved meanwhile.

\clearpage


\section*{Zusammenfassung}
(google translate version)

Die Ergebnisse von Neutrino-Oszillation Experimente deuten darauf hin, dass mindestens zwei Neutrinos Masse haben. Aber was die absolute Masse des Neutrinos Massstab ist und ob Neutrino und Anti-Neutrino sind ihre eigenen Anti-Teilchen nach wie vor ungel\"ost. Neutrinoless double beta decay Experimente zur Verbesserung unseres Verst\"andnisses der Probleme und sind die praktischste Methode, um die zweite Frage.

Die GERmanium Detector Array (GERDA) Experiment der Suche nach dem neutrinoless doppelte Beta-Zerfall von $^{76}$Ge ist derzeit im Aufbau in der Halle A des INFN Gran Sasso National Laboratory (LNGS), Italien. Um eine extrem niedrige Hintergrund Ebene, segmentierten Germanium-Detektoren werden direkt in fl\"ussigem Argon die als K\"uhl-und Abschirmmaterial gleichzeitig.

Mehrere Tests Kryostaten wurden f\"ur den Betrieb segmentierten Germanium-Detektoren im Vakuum und kryogene Fl\"ussigkeit auf dem Max-Planck-Institut f\"ur Physik in M\"unchen, Deutschland. Die Leistungsf\"ahigkeit und den Hintergrund der Diskriminierung Macht der segmentierten Germanium-Detektoren wurden im Detail untersucht. Es wurde bewiesen, für die das erste Mal, dass die segmentierten Germanium-Detektor kann direkt in kryogenen Fl\"ussigkeit stabil f\"ur eine lange Zeit, und dass die Segmentierung des Systems besch\"aftigt hat auch bei der Identifizierung von Photonen-und Neutronen-induzierten Hintergrund.

Eine umfassende C++-Framework, MaGe (Majorana-Gerda), wird gemeinsam von der Majorana und GERDA Kooperationen. Es basiert auf Geant4, aber abgestimmt auf die besonders geeignet für die Simulation der Reaktion der ultra-niedrige radioaktive Strahlung Hintergrund Detektoren durch ionisierende Strahlung. Die Vorhersagen der Simulationen wurden \"uberpr\"uft, um zu 5 \%.

Pulsform-Analyse ist eine erg\"anzende Methode zur Segmentierung für die weitere Ermittlung Hintergrund Veranstaltungen. Die Ermittlung der Effizienz muss anhand zuverl\"assiger Pulsform Simulationen. Eine vollst\"andige funktionelle Pulsform Simulation Paket wurde entwickelt, in der Magier Rahmen. Die Simulation wurde \"uberpr\"uft anhand der Daten aus den ersten Prototyp segmentierten Detektor f\"ur Gerda. Das Verst\"andnis f\"ur die Eigenschaften der segmentierten Germanium-Detektoren wurde inzwischen verbessert.

\clearpage

%%% Local Variables:
%%% mode:latex
%%% TeX-master: "thesis"
%%% End:
