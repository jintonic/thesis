\section*{Abstract} 
The results from neutrino oscillation experiments indicate that at
least two neutrinos have mass. However, what the absolute mass scale
for neutrinos is and whether neutrino and anti-neutrino are identical,
i.e. Majorana particles remains unknown.  Neutrinoless double beta
decay experiments can help to improve our understanding in both cases
and are the most practical method known to tackle the second question.
 
The GERmanium Detector Array (GERDA), searching for neutrinoless
double beta decay of $^{76}$Ge, is currently under construction in
Hall A of the INFN Gran Sasso National Laboratory (LNGS), Italy. In
order to achieve an extremely low background level, segmented
germanium detectors will be operated directly in liquid argon which
serves as cooling and shielding material simultaneously.
 
Several test cryostats were built to operate segmented germanium
detectors in vacuum and submerged in cryogenic liquid at the
Max-Planck-Institut f\"ur Physik in M\"unchen. The performance and the
background discrimination power of segmented germanium detectors were
studied in detail.  It was proved for the first time that segmented
germanium detectors can be operated stably over a long period
submerged in a cryogenic liquid.  It was confirmed that the
segmentation scheme employed does well in the identification of photon
induced background and demonstrated for the first time that also
neutron interactions can be identified.
 
 
The C++ Monte Carlo framework, MaGe (Majorana-Gerda), is a joint
development of the Majorana and GERDA collaborations.  It is based on
GEANT4, but tailored especially to simulate the response of ultra-low
background detectors to ionizing radiation.  The predictions of the
simulation were verified to hold to 5\,\%.
 
Pulse shape analysis is complementary to segmentation in identifying
background events.  Its efficiency can only be correctly determined
using reliable pulse shape simulations.  A fully functional pulse
shape simulation package was developed to augment the MaGe
package. The simulation was verified using data taken with the first
segmented prototype detector for GERDA. This work also led to a
considerable improvement in the understanding of segmented germanium
detectors.
 
\clearpage 
 
 
\section*{Zusammenfassung} 

 
Die Ergebnisse von Neutrinooszillationsexperimenten zeigen, dass
mindestens zwei Neutrinos eine endliche Masse haben.  Die absolute
Massenskala ist jedoch nicht bekannt. Ungel\"ost ist auch die Frage,
ob das Neutrino sein eigenes Antiteilchen ist, i.e.  ob Neutrinos
Majoranateilchen sind.  Eine m\"ogliche Beobachtung von neutrinolosem
Doppelbetazerfall k\"onnte zur Feststellung der Massenskala beitragen
und ist im Moment die einzige realisierbare M"oglichkeit, die Frage
nach der Natur der Neutrinos zu kl\"aren.
  
Das GERmanium Detector Array (GERDA) f\"ur die Suche nach
neutrinolosem Doppelbetazerfall von $^{76}$Ge wird derzeit in der
Halle A des ``INFN Gran Sasso National Laboratory (LNGS)'' in Italien
aufgebaut.  Um ein extrem niedriges Untergrundniveau zu ereichen,
werden segmentierte Germaniumdetektoren direkt in fl\"ussigem Argon,
das gleichzeitig als K\"uhl-und Abschirmmedium dient, betrieben.


Mehrere Testkryostaten wurden f\"ur den Betrieb segmentierter
Germaniumdetektoren in Vakuum und kryogener Fl\"ussigkeit am
Max-Planck-Institut f\"ur Physik in M\"unchen entwickelt.  Es wurde
zum ersten Mal gezeigt, dass segmentierte Germaniumdetektoren \"uber
lange Zeit stabil in einer kryogenen Fl\"ussigkeit betrieben werden
k\"onnen.  Die M\"oglichkeiten, Untergrundereignisse in segmentierten
Germaniumdetektoren zu identifizieren, wurden im Detail untersucht.
Dabei wurde betst\"atigt, dass geeignete Segmentierung die
Identifikation von photoninduzierten Ereignissen erm\"oglicht und zum
ersten Mal gezeigt, dass auch neutroninduzierte Ereignisse
identifiziert werden k\"onnen.

Das C++ Monte Carlo Paket MaGe (Majorana-Gerda) ist ein
Gemeinschaftprojekt der Majorana und GERDA Kollaborationen. Es basiert
auf GEANT4, ist aber zurechtgeschnitten auf die Simulation der
Wechselwirkungen ionisierender Strahlung mit Detektoren f\"ur
Experimente in einer Umgebung mit extrem niedriger Radioaktivit\"at.
Die Simulationen wurden mit Daten \"uberpr\"uft und beschreiben diese
mit maximalen Abweichungen von 5\,\%.
                                                                         

Pulsformanalyse kann Segmentierung bei der Identifikation von dem
Untergrund zuzurechnenden Ereignistopologien erg\"anzen.  Zur
Ermittlung der Effizienz von Pulsformanalysen wird eine exakte
Pulsformsimulation ben\"otigt.  Ein komplettes
Pulsformsimulationspaket wurde im Rahmen diesr Arbeit entwickelt.  Die
simulierten Pulse wurden mit Messungen am ersten segmentierten
Prototypdetektor verglichen und den Daten erfolgreich angepasst.
Dabei wurde auch das Verst\"andnis der Detektoren signifikant
verbessert.
 
\clearpage 
 
%%% Local Variables: 
%%% mode:latex 
%%% TeX-master: "thesis" 
%%% End: 
