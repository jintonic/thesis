%$Id$
Back to the days when the \emph{Standard Model} was established the
neutrino was believed to be massless based on the experimental
evidences that it had very tiny mass and always had the same
chirality. However, the story was changed dramatically because of the
observation of neutrino oscillations, which may be very naturally
explained if the weak interaction eigenstates of neutrinos are
admixture of their mass eigenstates and the latter propagate with
difference velocities. The introduction of mass terms of neutrinos
into the \emph{Standard Model} becomes necessary.

There are various methods to introduce the mass terms of neutrinos
into the \emph{Standard Model}. The most straightforward approach is
to follow the same way of introducing mass for the electron, that is,
the lepton obtains mass by coupling with Higgs. The problems of this
approach are, at first, it does not explain why neutrinos couple with
Higgs so weakly compared with their leptonic partners, and secondly,
it requires the introduction of right-handed neutrinos which are not
observed yet in the experiment. A very elegant way to solve these
problems is to assume that neutrinos are Majorana particles, that is,
their own anti-particles. This way, the second problem is solved
automatically.  And once the Majorana mass terms are introduced into
the Lagrangian the first problem can be solved by using \emph{see-saw
  mechanism}.

Physicists are investigating different methods to verify that
neutrinos are Majorana particles both theoretically and
experimentally. The most practical way for the time being is to search
for neutrinoless double beta decay. In this process, a neutrino
emitted from one beta decay is absorbed by another beta decay, which
can only happen if neutrinos are Majorana type. Ten naturally
occurring isotopes are observed to be capable of undergoing
double-beta decay. Among them, $^{76}Ge$ is of special importance
because, at first, germanium is a semiconductor material and can be
made into detectors with very high energy resolution, it can serve as
source and detector simultaneously, the efficiency of detection is
very high, secondly, germanium is the purest material that can be
produced in the world, the intrinsic background source is very
limited.


The GERDA (GERmanium Detector Array) experiment searching for
neutrinoless double beta decay of $^{76}Ge$ is currently under
construction in Hall A of the INFN Gran Sasso National Laboratory
(LNGS), Italy. In order to achieve extremely low background index,
18-fold segmented germanium detectors will be operated directly in
cryogenic liquid serving as cooling and shielding material in the
second phase of GERDA. The main purposes of this thesis under this
context are to examine systematically the operation and performance of
segmented detectors in cryogenic liquid, and to investigate their
power of background discrimination by analyzing the spatial
distribution over which energy is deposited and the time structure of
the detector response.

Several test facilities were built for the purposes described above.
\emph{Siegfried}, an 18-fold segmented prototype detector for GERDA
Phase II was operated in a traditional cryostat. The studies based on
this test stand are:
\begin{itemize}
\item the characterization of the prototype detector and its
  electronics, including the resolutions, cross talks, segment
  boundaries, the thickness of the dead layers, crystal axes and
  impurities, etc.
\item the analysis of background induced by external photons in
  MeV-energy range. These photons typically undergo multiple Compton
  scattering and deposit their energy over a range of several
  centimeters. This distinguishes them from the neutrinoless double
  beta decay signals which deposit energy on a millimeter scale.
\item the analysis of background induced by neutron interactions with
  germanium isotopes and surrounding materials. Most of the neutron
  induced events deposit energy in different segments of the detector.
  Particularly, the neutron inelastic scattering with germanium
  isotopes is of great interesting because of the entanglement of the
  nuclear recoil energy and the prompt photon energy.
\end{itemize}
\emph{Gerdalinchen II}, a specially designed cryostat containing
liquid nitrogen or argon, inside which different types of detectors
were operated, was used to carry out the following studies:
\begin{itemize}
\item the operation of several segmented detectors submerged directly
  in cryogenic liquid. Detailed operating procedures were
  investigated. The performance of the detectors were carefully
  monitored and analyzed.
\item the screening of the prototype detectors, the data from which
  were used to study the time structure of the detector response, i.e.
  pulse shape. A reliable pulse shape simulation package was developed
  and verified by being compared with data.
\item two more neutron experiments with better shielding. The
  influence of neutron interactions with cryogenic liquid on the
  detector was examined. The Monte Carlo simulation of low energy
  neutron interactions was verified in detail.
\end{itemize}
Different pulse shape analyses were carried out based on the data from
different test stands and the simulation.

The contents of the thesis are summarized as following:
\begin{description}
\item[Chapter 2] describes the theoretic background of neutrinoless
  double beta decay, and other approaches to check whether neutrinos
  are Majorana type or Dirac type.
\item[Chapter 3] introduces the basic ideas of the GERDA experiment,
  summarizes the latest results of neutrinoless double beta decay from
  previous experiments, compares GERDA with its competitive
  experiments and estimates the potential of the future neutrinoless
  double beta decay experiments.
\item[Chapter 4] summarizes the basic concepts of semiconductor
  detectors and the important properties of germanium crystals and
  detectors related to the later analysis
\item[Chapter 5] introduces the two test stands that all the analysis
  is based on, describes the slow control and data acquisition system
  relying on which the test stands were running.
\item[Chapter 6] characterizes the short and long term performance of
  segmented germanium detectors in cryogenic liquid test stand.
\item[Chapter 7] demonstrates how good the segmented detectors are to
  identify different kinds of background, especially introduced by
  neutron interactions with germanium isotopes and surrounding
  materials.
\item[Chapter 8] describes the methods to simulate the pulse shapes of
  different types of interactions in germanium detectors, verifies the
  simulation by comparing it with the measurements.
\item[Chapter 9] further classifies the background events by using
  different pulse shape analysis methods , compares them to each
  other, and investigates the power of integrating pulse shape
  analysis with the analysis based on the detector segmentation.
\item[Chapter 10] summarizes all the studies in the concept of the
  GERDA Phase II experiment, discusses the meanings of the studies for
  GERDA, and gives a outlook on further studies.
\end{description}

%%% Local Variables:
%%% mode:latex
%%% TeX-master: "thesis"
%%% End:
