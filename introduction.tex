\chapter{Introduction}
\label{cha:intro}
At the time the \emph{Standard Model} was established the neutrino was believed to be massless. In experiments it always had the same chirality and there was no evidence for a non-zero mass. However, the picture changed dramatically when neutrino oscillations were observed in solar and atmospheric neutrinos. They are explained by the weak interaction eigenstates of neutrinos being admixture of mass eigenstates and the latter propagating with different velocities. The introduction of neutrino mass terms into the Standard Model becomes necessary.

There are various methods to introduce the mass terms of neutrinos into the Standard Model. The most straightforward approach is to follow the same procedure as for the electron, \textit{i.e.} the lepton obtains mass by coupling to the Higgs field. The problems of this approach are, that it does not explain why neutrinos couple to the Higgs field so weakly compared to their leptonic partners, and that it requires the introduction of right-handed neutrinos which are not yet experimentally observed. An elegant way to solve these problems is to assume that neutrinos are Majorana particles, \textit{i.e.} their own anti-particles. This way, the second problem does not arise, and once the Majorana mass terms are introduced into the Lagrangian the so-called \emph{see-saw mechanism} can make the different coupling strengths look natural.

Different theoretical and experimental methods are under investigation to verify that neutrinos are Majorana particles. The only experimental test of a Majorana nature of the neutrino currently possible is neutrinoless double beta ($0\nu\beta\beta$) decay. In this process, a neutrino emitted from one beta decay is absorbed by another beta decay. This can only occur, if neutrinos are of Majorana type. About ten naturally occurring isotopes are observed to have double-beta decay. Among them $^{76}Ge$ is of special importance because germanium is a semiconductor material used in high efficient detectors with very good energy resolution (it can serve as source and detector simultaneously), and it is the purest material produced in the world limiting intrinsic background.

The GERDA (GERmanium Detector Array) experiment \cite{Abt04, Sch05} searching for $0\nu\beta\beta$ decay of $^{76}Ge$ is currently under construction in Hall A of the INFN Gran Sasso National Laboratory (LNGS), Italy. In order to achieve an extremely low background index in the second phase of GERDA 18-fold segmented germanium detectors will be operated directly in cryogenic liquid serving as cooling and shielding material. The main goal for the work presented in this thesis is to examine systematically the operation and performance of segmented detectors in cryogenic liquid, and to investigate their power of background discrimination by analyzing the spatial distribution over which energy is deposited and the time structure of the detector response.

Several test facilities were built:\\
\emph{Siegfried~I}, an 18-fold segmented prototype detector for GERDA Phase~II was operated in a vacuum cryostat. This allowed
\begin{itemize}
\item the characterization of the prototype detector, including the segment boundaries, crystal axes and impurities, \textit{etc.}. The data was used to study the time structure of the detector response, \textit{i.e.} pulse shape. 
\item the analysis of background induced by external photons in the MeV-energy range. These photons typically undergo multiple Compton scattering and deposit their energies over a range of several centimeters. This distinguishes them from the electrons from $0\nu\beta\beta$ decay which deposit energy on a millimeter scale.
\item the analysis of background induced by neutron interactions with germanium isotopes and surrounding materials. Most of the neutron induced events deposit their energies in different segments of the detector. Particularly, the neutron inelastic scattering on germanium isotopes is of great interest because of the entanglement of the nuclear recoil energy and the prompt photon energy.
\end{itemize}
\emph{Gerdalinchen~II}, a specially designed cryostat containing liquid nitrogen or argon, inside which up to three segmented detectors could be operated simultaneously, was used to try out for the first time the operation of segmented detectors submerged directly in cryogenic liquid. Detailed operating procedures were investigated. The performance of the detectors was carefully monitored and analyzed.

The test facilities were modeled within a Geant4 based simulation framework, MaGe, which is jointly developed by GERDA and Majorana collaborations. The measurements mentioned were simulated using MaGe. The simulations of low energy (in the order of MeV) electron, photon and neutron interactions with germanium detectors and surrounding materials were verified in detail. A full-functional pulse shape simulation package was also developed within the MaGe framework. The whole signal formation process in the segmented germanium detector system was simulated and verified by being compared with data. 

The content of the thesis is summarized as follows:
\begin{description}
\item[Chapter~\ref{cha:theory}] describes the theoretic background of $0\nu\beta\beta$ decay, and other approaches to check whether neutrinos are of Majorana or Dirac type.
\item[Chapter~\ref{cha:exps}] summarizes different technical approaches of searching for $0\nu\beta\beta$ decays from different isotopes, compares the competitive experiments with each other and estimates the potential of future $0\nu\beta\beta$ decay experiments.
\item[Chapter~\ref{cha:gerda}] introduces the basic ideas of the GERDA experiment, summarizes the latest progress, and estimates the observation potential of GERDA.
\item[Chapter~\ref{cha:detector}] describes the basic concepts of semiconductor detectors and the important properties of germanium crystals and detectors related to the later analysis.
\item[Chapter~\ref{cha:teststand}] introduces the two test stands that provided the data for all studies, describes the slow control and data acquisition system relying on which the test stands were running.
\item[Chapter~\ref{cha:GII}] characterizes the short and long term performance of segmented germanium detectors in cryogenic liquid test stand.
\item[Chapter~\ref{cha:photon}] demonstrates the power of the segmented detectors to identify photon induced background, verifies the predictions of the simulation.
\item[Chapter~\ref{cha:neutron}] demonstrates the power of the segmented detectors to identify neutron interactions with germanium isotopes and surrounding materials, validates the simulation in this aspect as well.
\item[Chapter~\ref{cha:pss}] describes the physics models of the charge carrier drift inside germanium detectors, introduces methods to convolve the electronic effects in the pulse shape simulation. 
\item[Chapter~\ref{cha:psa}] verifies the pulse shape simulation by comparing it to the data taken with the GERDA prototype detector, introduce new methods to determine the crystal orientation and impurity distributions.
\item[Chapter~\ref{cha:con}] summarizes all the studies in the concept of the GERDA Phase II experiment, discusses the meanings of the studies for GERDA, and gives a outlook on further studies.
\end{description}

%%% Local Variables:
%%% mode:latex
%%% TeX-master: "thesis"
%%% End:
