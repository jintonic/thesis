\addcontentsline{toc}{chapter}{Introduction}
\chapter*{Introduction}
\label{cha:intro}
At the time the \emph{Standard Model} was established the neutrino was believed to be massless. In experiments it always had the same chirality and there was no evidence for a non-zero mass. However, the picture changed dramatically when neutrino oscillations were observed in solar and atmospheric neutrinos. They were explained by the weak interaction eigenstates of neutrinos being admixture of mass eigenstates and the latter propagating with different velocities. The introduction of neutrino mass terms into the Standard Model became necessary.

There are various methods to introduce neutrino mass terms into the Standard Model. The most straightforward approach is to follow the same procedure as for the charged leptons, \textit{i.e.} the leptons obtain mass by coupling to the Higgs field. The problems of this approach are, that it does not explain why neutrinos couple to the Higgs field so weakly compared to their charged partners, and that it requires the introduction of right-handed neutrinos which have not yet been experimentally observed. An elegant way to solve these problems is to assume that neutrinos are Majorana particles, \textit{i.e.} their own anti-particles. This way, the second problem does not arise, and once the Majorana mass terms are introduced into the Lagrangian, the so-called \emph{see-saw mechanism} can make the different coupling strengths look natural.

Different theoretical and experimental methods are under investigation to verify that neutrinos are Majorana particles. The only experimental test currently possible is the search for the neutrinoless double beta ($0\nu\beta\beta$) decay. In this process, a neutrino emitted from one beta decay is absorbed by another beta decay. This can only occur, if neutrinos are of Majorana type. About ten naturally occurring isotopes are observed to undergo double-beta decay. Among them, $^{76}Ge$ is of special importance because germanium is a semiconductor material used in highly efficient detectors with very good energy resolution (it can serve as source and detector simultaneously), and it is the purest material produced in the world limiting intrinsic background.

The GERDA (GERmanium Detector Array) experiment \cite{Abt04, Sch05}, searching for the $0\nu\beta\beta$ decay of $^{76}Ge$, is currently under construction in Hall A of the INFN Gran Sasso National Laboratory (LNGS), Italy. In order to achieve an extremely low background level, the second phase of GERDA features 18-fold segmented germanium detectors operated directly in cryogenic liquid, serving as cooling and shielding material. The main goals of the work presented in this thesis are to examine systematically the operation and performance of segmented detectors in cryogenic liquid and to investigate their power of background discrimination by analyzing the spatial distribution over which energy is deposited. The time structure of the detector response is studied and simulated to lay the foundation for its use in background suppression.

Two test facilities were used to take data with two 18-fold segmented GERDA Phase~II prototype detectors:\\
\emph{Gerdalinchen~II}, a specially designed cryostat containing liquid nitrogen or argon, inside which up to three segmented detectors can be operated simultaneously. It was used 
\begin{itemize}
\item to operate and study for the first time segmented detectors submerged directly in cryogenic liquid.
\item to develop detailed operating procedures.
\item to carefully study the data of a prototype detector.
\end{itemize}
\emph{ A vacuum cryostat}, especially equipped to operate one segmented detector. It allowed
\begin{itemize}
\item the analysis of background induced by external photons in the MeV-energy range. These photons typically undergo multiple Compton scattering and deposit their energies over a range of several centimeters. This distinguishes them from the electrons from $0\nu\beta\beta$ decay which deposit energy on a millimeter scale.
\item the analysis of background induced by neutron interactions with germanium isotopes and surrounding materials. Most of the neutron induced events deposit their energies in different segments of the detector. Particularly, the inelastic scattering of neutrons at germanium isotopes can be identified through the separation of the energies deposited by the prompt photon and the nuclear recoil.
\item the characterization of the detector, including the segment boundaries, crystal axes and impurities, \textit{etc.}. The data was used to study the time structure of the detector response, \textit{i.e.} pulse shape and verify corresponding simulations. 
\end{itemize}

The test facilities were modeled using the a Geant4 based simulation framework, MaGe, which is jointly developed by the GERDA and Majorana collaborations. The measurements mentioned were simulated using MaGe. The simulation of low energy electrons, photons and neutrons interacting with germanium detectors and surrounding materials were verified in detail. A fully functional pulse shape simulation package was also developed within the MaGe framework. The whole signal formation process in segmented germanium detectors and the read-out system was simulated and verified by being compared with data. 

The thesis is structured as follows:
\begin{description}
\item[Chapter~\ref{cha:theory}] describes the theoretic background of $0\nu\beta\beta$ decay as well as other approaches to test whether neutrinos are of Majorana or Dirac type.
\item[Chapter~\ref{cha:exps}] summarizes different technical approaches of searching for $0\nu\beta\beta$ decays of different isotopes, compares the experiments with each other and estimates the potential of future $0\nu\beta\beta$ decay experiments.
\item[Chapter~\ref{cha:gerda}] introduces the basic ideas of the GERDA experiment, summarizes the latest progress, and estimates the potential of GERDA.
\item[Chapter~\ref{cha:detector}] describes the basic concepts of semiconductor detectors and the important properties of germanium crystals and detectors related to the later analysis.
\item[Chapter~\ref{cha:teststand}] introduces the two test stands that provided the data for all studies, describes the slow control and data acquisition system on which the test stands rely.
\item[Chapter~\ref{cha:GII}] characterizes the short and long term performance of segmented germanium detectors submerged in cryogenic liquid.
\item[Chapter~\ref{cha:np}] describes a special class of events with negative baseline shifts.
\item[Chapter~\ref{cha:photon}] demonstrates the power of segmented detectors to identify photon induced background and verifies the Monte Carlo simulation.
\item[Chapter~\ref{cha:neutron}] demonstrates the power of segmented detectors to identify neutron interactions with germanium isotopes and surrounding materials, validates the simulation in this aspect as well.
\item[Chapter~\ref{cha:pss}] describes the physics models of the charge carrier drift inside germanium detectors to simulate the pulse shape and introduces methods to add electronic effects.
\item[Chapter~\ref{cha:psa}] verifies the pulse shape simulation by comparing it to the data taken with the GERDA prototype detector and introduces new methods to determine the crystal orientation and impurity distributions.
\end{description}
The results are summarized within the context of the GERDA Phase~II experiment and an outlook to further studies is given.


%%% Local Variables:
%%% mode:latex
%%% TeX-master: "thesis"
%%% End:
