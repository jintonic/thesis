%$Id$
At the time the \emph{Standard Model} was established the neutrino was
believed to be massless. In experiments it always had the same
chirality and there was no evidence for a non-zero mass. However, the
picture changed dramatically when neutrino oscillations were observed
in solar and atmospheric neutrinos. They are explained by the weak
interaction eigenstates of neutrinos being admixture of mass
eigenstates and the latter propagating with different velocities. The
introduction of neutrino mass terms into the \emph{Standard Model}
becomes necessary.

There are various methods to introduce the mass terms of neutrinos
into the \emph{Standard Model}. The most straightforward approach is
to follow the same procedure as for the electron, \textit{i.e.} the
lepton obtains mass by coupling to the Higgs field. The problems of
this approach are, that it does not explain why neutrinos couple to
the Higgs field so weakly compared with their leptonic partners, and
that it requires the introduction of right-handed neutrinos which are
not observed yet in the experiment. An elegant way to solve these
problems is to assume that neutrinos are Majorana particles,
\textit{i.e.} their own anti-particles. This way, the second problem
does not arise, and once the Majorana mass terms are introduced into
the Lagrangian the so-called \emph{see-saw mechanism} can make the
different coupling strengths look natrual.

Different theoretical and experimental methods are under investigation to verify that neutrinos are Majorana particles. The only experimental evidence for a Majorana nature of the neutrino for the time being would be neutrinoless double beta decay ($0\nu\beta\beta$). In this process, a neutrino emitted from one beta decay is absorbed by another beta decay. This can only happen if neutrinos are of Majorana type. Ten naturally occurring isotopes are observed to have double-beta decay. Among them $^{76}Ge$ is of special importance because, at first, germanium is a semiconductor material and can be made into detectors with very good energy resolution, it can serve as source and detector simultaneously, the efficiency of detection is very high, secondly, germanium is the purest material that can be produced in the world, the intrinsic background source is very limited.

The GERDA (GERmanium Detector Array) experiment searching for
$0\nu\beta\beta$ decay of $^{76}Ge$ is currently under
construction in Hall A of the INFN Gran Sasso National Laboratory
(LNGS), Italy. In order to achieve an extremely low background index
in the second phase of GERDA 18-fold segmented germanium detectors
will be operated directly in cryogenic liquid serving as cooling and
shielding material. The main goal of this thesis is to examine
systematically the operation and performance of segmented detectors in
cryogenic liquid, and to investigate their power of background
discrimination by analyzing the spatial distribution over which energy
is deposited and the time structure of the detector response.

Several test facilities were built:\\
\emph{Siegfried}, an 18-fold segmented prototype detector for GERDA
Phase II was operated in a traditional cryostat. This allowed
\begin{itemize}
\item the characterization of the prototype detector and its
  electronics, including the resolutions, cross talks, segment
  boundaries, crystal axes and impurities, \textit{etc}.
\item the analysis of background induced by external photons in the
  MeV-energy range. These photons typically undergo multiple Compton
  scattering and deposit their energy over a range of several
  centimeters. This distinguishes them from the electrons from
  $0\nu\beta\beta$ decay which deposit energy on a millimeter
  scale.
\item the analysis of background induced by neutron interactions with
  germanium isotopes and surrounding materials. Most of the neutron
  induced events deposit energy in different segments of the detector.
  Particularly, the neutron inelastic scattering on germanium isotopes
  is of great interest because of the entanglement of the nuclear
  recoil energy and the prompt photon energy.
\end{itemize}
\emph{Gerdalinchen II}, a specially designed cryostat containing
liquid nitrogen or argon, inside which different types of detectors
were operated, was used to carry out the following studies:
\begin{itemize}
\item the operation of several segmented detectors submerged directly
  in cryogenic liquid. Detailed operating procedures were
  investigated. The performance of the detectors was carefully
  monitored and analyzed.
\item the scanning of the prototype detectors. This data was used to
  study the time structure of the detector response, \textit{i.e.}
  pulse shape. A reliable pulse shape simulation package was developed
  and verified by being compared with data.
\item two neutron experiments with improved shielding. The influence
  of neutron interactions with cryogenic liquid on the detector was
  examined. The Monte Carlo simulation of low energy neutron
  interactions was verified in detail.
\end{itemize}
Different pulse shape analyses were carried out based on the data from
different test stands and the simulation.

The content of the thesis is summarized as following:
\begin{description}
\item[Chapter 2] describes the theoretic background of   $0\nu\beta\beta$ decay, and other approaches to check whether   neutrinos are of Majorana type or Dirac type.
\item[Chapter 3] introduces the basic ideas of the GERDA experiment,
  summarizes the latest results of $0\nu\beta\beta$ decay from
  previous experiments, compares GERDA with competitive experiments
  and estimates the potential of future $0\nu\beta\beta$ decay
  experiments.
\item[Chapter 4] summarizes the basic concepts of semiconductor
  detectors and the important properties of germanium crystals and
  detectors related to the later analysis
\item[Chapter 5] introduces the two test stands that provided the data
  for all studies, describes the slow control and data acquisition
  system relying on which the test stands were running.
\item[Chapter 6] characterizes the short and long term performance of
  segmented germanium detectors in cryogenic liquid test stand.
\item[Chapter 7] demonstrates the power of the segmented detectors to
  identify different kinds of background, especially introduced by
  neutron interactions with germanium isotopes and surrounding
  materials.
\item[Chapter 8] describes the methods to simulate the pulse shapes of
  different types of interactions in germanium detectors, verifies the
  simulation by comparing it with the measurements.
\item[Chapter 9] further classifies the background events by using
  different pulse shape analysis methods , compares them to each
  other, and investigates the power of integrating pulse shape
  analysis with the analysis based on the detector segmentation.
\item[Chapter 10] summarizes all the studies in the concept of the
  GERDA Phase II experiment, discusses the meanings of the studies for
  GERDA, and gives a outlook on further studies.
\end{description}

%%% Local Variables:
%%% mode:latex
%%% TeX-master: "thesis"
%%% End:
