\chapter{Summary and outlook} 
\label{cha:con} 
 
The results presented in this thesis can be summarized as follows:
\begin{enumerate}
\item Segmented $n$-type germanium detectors can be operated stably
over long periods submerged in a cryogenic liquid;
\item They can facilitate the discrimination of both photon- and
neutron-induced background from beta decay signals.
\item The pulse shapes of segmented detectors can be simulated
reliably using basic information about semiconductor-detectors;
\item A novel way to determine the crystal orientation and impurity
distribution of a segmented detector was developed.
\end{enumerate}

The work was performed in the context of detector development for
Phase~II of the GERDA neutrinoless double beta decay experiment. The
results are highly relevant for the realization of this experiment.

\begin{enumerate}

\item Segmented $n$-type germanium detectors are considered to be used
in GERDA Phase~II which is based on the idea to submerge detectors in
liquid argon to achieve extremely low background levels using the
liquid as a shield against external radiation.
\item The background level due to the predicted radioactivity within
GERDA can only reach the extremely low level targeted, if both
photon- and neutron-induced backgorund can be rejected with good
efficiency.
\item Even though background events are identified very well using
segment information, an additional suppression factor of
$\approx$\,1.3 for photon-induced events is expected from pulse shape
analysis. However, the verification of this expectation is important
and impossible without excellent pulse shape simulation.
\item As handling has to be minimized during the production of the
GERDA detectors in order to minimize possible contamination, the
control measurements should be kept to a minimum. This novel way to
determine the crystal orientation needed for pulse shape analysis
allows an in-situ measurement of crystal properties during a normal
energy calibration inside the GERDA cryostat.
\end{enumerate}

As part of the work presented in this thesis, many tests were made to
study segmented germanium detectors and the complete read-out and
support system.  The understanding of these novel kinds of detectors
was enhanced considerably. The operation of the detectors and the
tuning of the electronics provides guidelines for the operation of
GERDA. This includes monitoring and control software.

The Monte Carlo package, MaGe, used to simulate all aspects of the
GERDA experiment, was verified using photon-induced and
neutron-induced events. Several problems, especially in the simulation
of neutron interactions, were identified and solved. The overall
agreement of the Monte Carlo predictions with data taken in the test
facilities was very satisfactory.

The detector system including read-out and signal transmission and
processing is still being optimized:
\begin{itemize} 
\item the signal transmission inside the cryogenic liquid needs to be
further improved to minimize cross-talks and micro-phonic effects;
\item the high voltage distribution into the cryogenic volume has to
be improved to allow stable running not only with liquid nitrogen, but
also with liquid argon;
\end{itemize}

It is planned to operate three segmented germanium detectors together
in liquid argon, providing a test environment as close as possible to
the GERDA environment and allowing a large variety of detector
studies.

\begin{itemize}
\item Data will be collected with a low energy gamma source placed
inside the core of a segmented $n$-type detector. Holes created close
to the inner surface of the detector will drift outwards while
electrons reach the inner surface almost immediately.  Thus, the drift
of the holes can be studied separately.
\item Various surface scans will be performed on at least three
different segmented detectors of the same type. This will allow the
verification of the pulse shape simulation package with a large
variety of data.
\item Several pulse shape analyses will be performed and evaluated
using data and Monte Carlo including pulse shape simulation. The
potential for background discrimination will be studied in detail.
\end{itemize} 
 
All these studies will rely on the work presented in this thesis and
will further the understanding of segmented germanium detectors. The
results will provide information for the operation of GERDA and the
planning of future germanium based experiments.
 
%%% Local Variables: 
%%% mode:latex 
%%% TeX-master: "thesis" 
%%% End: 
