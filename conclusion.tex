\chapter{Conclusions and outlook}
\label{cha:con}
The neutrinos in Standard Model are massless particles; they participate in the weak interaction with fixed helicity; there are only left-handed neutrinos and right-handed anti-neutrinos, and they are assumed to be different particles. The observation of neutrino oscillations indicated that at least two neutrino mass eigenstates have non-zero mass. This makes it necessary to introduce neutrino mass terms into the Standard Model Lagrangian. 

Following the same procedure as for the electron Dirac mass terms of neutrinos can be introduced. However, this does not explain why neutrinos couple to the Higgs field so weakly compared to their leptonic partners, and why the right-handed neutrinos and left-handed anti-neutrinos have never been found so far. Being ware of the chargeless nature of neutrinos the charge conjugates of the neutrinos fields can be used to form Majorana mass terms. Neutrinos and anti-neutrinos are assumed to be identical in this scenario, hence the second problem does not arise. The see-saw mechanism taking into account both Dirac and Majorana mass terms can be used to the different coupling strengths look natural.

The absolute mass scale of neutrinos cannot be measured by the oscillation experiments. It is constrained by cosmological observations, single beta decay and neutrinoless double beta ($0\nu\beta\beta$) decay experiments. Instead of probing neutrino masses, $0\nu\beta\beta$ decay experiments are the only approach known to answer the question whether neutrino and anti-neutrino are identical or not, hence were and are widely carried out. There was one controversial observation claim made by part of the Heidelberg-Moscow collaboration recently. The corresponding effective Majorana neutrino mass was also given by them. It is of great interest to perform new experiments with higher sensitivities to verify their result.

Many experiments have been constructed or planned to search for $0\nu\beta\beta$ decays of different isotopes. Generally speaking, large amount of source material, high abundance of the isotopes under study, large signal detecting efficiency, good energy resolution and low background contamination are preferred in order to overcome the extremely low event rate of the $0\nu\beta\beta$ decay. There is no simple solution satisfying all the preference. Different technical approaches were compared to each other. Their Pros and Cons were summarized.

The experiments searching for the $0\nu\beta\beta$ decay of $^{76}$Ge are able to take advantage of the excellent energy resolution of germanium detectors to minimize the contamination of neutrino accompanied double beta $2\nu\beta\beta$ decay. Since the source and detector are identical, the detecting efficiency is very high. And The low natural aboundance of $^{76}$Ge can be overcome by entrichment. However, this approach suffers from the fact that the $Q$-value of the decay is only $2039$~keV, lower than many natural radiation lines. To effectively reduce the background contamination is the key for this kind of experiment to success.

The GERmanium Detector Array (GERDA) experiment, currently under construction in Hall A of the INFN Gran Sasso National Laboratory (LNGS), Italy, follows this approach. 

%%% Local Variables:
%%% mode:latex
%%% TeX-master: "thesis"
%%% End:
