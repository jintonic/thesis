When particles, $\alpha, \beta, \gamma, n, p$, etc., interact inside the germanium semiconductor detector, they will create electron-hole pairs, which act as charge carriers. Due to the electric field inside the detector the charge carriers drift and induce electric signals in electrodes. The electric signals are amplified, digitized and recorded by the electronics and data acquisition systems connected to the germanium detector. The whole process of signal formation in germanium detectors and their surrounding electronics is briefly reviewed in this chapter.

\section{Interactions of radiation with matter}
\label{sec:det:phys}
\subsection{Electron and positron}
\label{sec:det:ep}
Electrons and positrons traversing matter lose their kinetic energy mainly by two processes, ionization and bremsstrahlung. High energy (GeV range) electrons and positrons predominantly lose energy by bremsstrahlung. Low energy (MeV range) electrons and positrons predominantly lose energy by ionization. The energy at which an electron or a positron loses as much energy in collisions as in radiation is called critical energy $\epsilon$. For elements with charge $Z > 13$ the critical energy is~\cite{Ama81}
\begin{equation}
  \label{eq:det:ecrit}
  \epsilon = (550/Z) \text{ MeV}.
\end{equation}
For germanium $\epsilon \approx 17$~MeV. Hence, electrons emitted from the double beta decay of $^{76}$Ge mainly loss their energy by ionization.

The range of electrons and positrons depends on their energy and the material they traverse (see Ref.~\cite{Bri84} and references therein). The average range of a 1~MeV electron in germanium is about 0.5~mm.

When a positron loses all its kinetic energy, it annihilates with an electron into two photons with an energy of 511~keV each, corresponding to the rest mass of electrons.

\subsection{Photon}
\label{sec:det:gamma}
Photons emitted by natural radioactive isotopes have energies ranging from several keV to several MeV. The possible interaction processes of photons in matter are photoelectric effect, Compton (incoherent) scattering, Rayleigh (coherent) scattering and pair production. Since Rayleigh scattering does not change the energy of the scattered photon but only its momentum, it is not relevant for GERDA.

\textbf{Photoelectric effect:} When a photon interacts with an atom, its entire energy may be transfered to an atomic shell electron which is kicked out of the shell. The energy of the secondary electron is equal to the difference of the incident photon energy and the binding energy of the electron. If the electron is kicked out from an inner shell of the atom, an outer shell electron fills the vacancy. Consequently, either characteristic x-rays are emitted, or, if the x-ray photons are re-absorbed, secondary Auger-electrons are emitted. The cross section of the photoelectric effect is inversely proportional to the photon energy. Hence, the photoelectric effect is the main interaction mechanism for photons at low energy (up to about 200~keV for Ge).

\textbf{Compton scattering:} A photon scatters on a weakly bound electron (quasi-free), transferring only part of its energy and momentum. The angular distribution of the scattered photon is described by the Kein-Nishina formula. The maximum energy transfer occurs when the incident photon is scattered by $180^{\circ}$. Compton scattering is the predominant interaction process for photons of energies between about 200~keV and 8~MeV for germanium. A 1.33~MeV photon undergoes on average three Compton scatterings before being absorbed through the photoelectric effect. The mean free path of such a photon is about three centimeters.

\textbf{Pair production:} If the photon energy exceeds twice the rest mass of an electron, the photon can create an electron-positron pair in the electric field of a nucleus, the rest of the photon energy being transferred to the created electron and positron as kinetic energy. A significant cross section for this interaction mechanism arises only for energies above $4\sim5$~MeV.

\subsection{Neutron}
\label{sec:det:neutron}
Because of the lack of charge, neutrons have a relatively high penetration power. However, there are still five processes that occur when neutrons interact with nuclei depending on the kinetic energy of incident neutrons:

\textbf{Capture:} The nucleus, Z, absorbs the incident neutron, $n$, and de-excites with the emission of one or more photons, noted as $^{\text{A}}$Z$(n,\gamma)$, where A is the atomic number of the nucleus. In case of \textbf{internal conversion} an electron from a lower shell of the nucleus is emitted instead of a photon, noted as $^{\text{A}}$Z$(n,e)$. The excited nucleus can also be meta-stable and not de-excite instantaneously, noted as $^{\text{A}}$Z$(n,\gamma^{m})$. Capture is the dominant process for thermal neutrons, \textup{i.e.}, neutrons with energies in the sub-eV range.

\textbf{Elastic scattering:} A neutron collides with a nucleus, transfers some energy to it and bounces off in a different direction; the target nucleus gains the energy lost by the neutron (also called recoil energy). This is one of the significant processes for neutrons with energies in the range of keV to several tens of MeV. The recoil energy is very small (mostly less than 200~keV) and has an exponentially decaying distribution, hence not relevant as a background for $0\nu\beta\beta$ decay search.

\textbf{Inelastic scattering:} A nucleus temporarily absorbs the incident neutron, forming a compound nucleus in an excited state; it then de-excites by emitting another neutron of lower energy, together with a photon, which takes the de-excitation energy; the nucleus also takes some recoil energy. The reaction is noted as $^{\text{A}}$Z$(n,n^{\prime}\gamma)$ and is another significant process for neutrons with energy in the range of 1 to several tens of MeV.

\textbf{Transmutation:} A nucleus absorbs a neutron then de-energizes by emitting a proton, $p$, or an $\alpha$ particle. This produces a nucleus of a different element. The process occurs when the incident neutron energy exceeds a hundred MeV and becomes dominant at several GeV.

When the neutron energy becomes even higher, fission reactions occur. High energy neutrons can only be produced by cosmic ray muons which are vetoed by the muon detecting system in GERDA. However, the meta-stable nuclei excited by such neutrons could be a serious background for $0\nu\beta\beta$ decay search (see Sec.\ref{sec:gerda:santi}).


\section{Germanium detectors}
\label{sec:det:semi}
\subsection{Working principle of semiconductor detectors}
\label{sec:det:prin}
Insulators, semiconductors and conductors are distinguished according to the energy gap between the valence and the conduction band. The widths of the band gap of semiconductors are between those of insulators and conductors and are of the order of several eV. This allows an electron to be lifted from the valence band to the conduction band by thermal motion or external ionizing radiation.

The resistivity of a semiconductor can be modified by impurities (dopant) in its crystal lattice. Elements with only three valence electrons create energy states a little bit higher than the valence band, making it easy to lift valence electrons to these states and create freely moving holes in the valence band. This kind of dopant is called acceptor. Elements with five valence electrons donate electrons in the energy states a little bit lower than the conduction band, making it easy to lift these weakly bound electrons to the conduction band. This kind of dopant is called donor. Usually the thermal energy available at room temperature is sufficient to ionize most of the dopant. The created freely moving electrons and holes are called charge carriers.

A p-n junction forms when p- and n-doped pieces of semiconductor are placed together. The majority charge carriers diffuse into regions with lower concentrations, and are eliminated by recombination, leaving behind the charged ions adjacent to the interface in a region with no mobile carriers called the \emph{depletion zone}. Since these ions create positive space charges on the n side and negative ones on the p side, an electric field is created providing a force opposing the continued diffusion of charge carriers. The size of the depletion zone changes when an external potential is applied to the junction. Under \emph{reverse bias} (anode connected to the n side, cathode to the p side) the majority charge carriers are driven away from the junction. This widens the depletion zone. Since the carrier density is small in the depletion zone, only a very small reverse saturation current flows through. Likewise, the depletion zone squeezes under \emph{forward bias} while the current greatly increases.

The depletion zone is the active volume of any semiconductor detector. When ionizing radiation strikes the depletion zone, some of its deposited energy excites electrons out of the valence band and electron-hole pairs are created. Due to the electric field electron-hole pairs cannot recombine but split and drift towards the electrodes and induce electric signals there. In most of the cases a reverse bias is applied in order to create an active volume as large as possible. The bias voltage turning the whole bulk of a semiconductor detector into a depletion zone is called the full depletion voltage.

\subsection{Operating voltage of germanium detectors}
\label{sec:det:volt}
The depletion depth is proportional to the square root of the ratio of the applied voltage and the dopant concentration. The fewer dopants, the bigger the depleted region under a certain reverse bias. The number of active impurities per cubic centimeter in germanium crystals is of the order of $10^{10}$. The number of germanium atoms in the same volume is of the order of $10^{22}$. So the dopant concentration in germanium detectors is of the order of 1~ppt (particle per trillion) which is extremely low. Consequently, the depletion voltages of germanium detectors are two orders of magnitude lower than for silicon detectors of the same size. This allows the construction of large germanium detectors operating at relatively low voltage. The largest germanium detectors are based on a cylindrical geometry with diameter and height both in the several centimeter range. The full depletion voltage for them is just several kilovolts. The operating voltage is normally a little bit higher than that.

\subsection{Operating temperature of germanium detectors}
\label{sec:det:temp}
The smaller the band gap, the higher the probability that an electron is transferred to the conduction band. The band gap in germanium is 0.72~eV, in silicon it is 1.1~eV. At room temperature the population of electrons in the conduction band in germanium is a factor of 1000 higher than in silicon. Applying a bias voltage to a germanium detector with a large bulk (about several cm) at room temperature would create a large current. This would make the operation impossible\footnote{The large bulk current creates large noise   distorting any signal.}, or even destroy the detector. Germanium detectors are usually cooled via a metal cooling stick submerged in cooling medium, \textit{e.g.}, liquid nitrogen, to suppress thermal excitation.\footnote{The number of thermally excited electrons could   be very small when the germanium volume becomes small.  Germanium   pieces thinner than several microns are used to make room   temperature detectors. No high voltage is needed to create the   depletion zone. The leakage current is very small. The noise from   the thermal excitation also becomes tolerable given very few   thermally excited electrons.} Due to imperfect heat conduction the temperature of the germanium crystal is slightly higher than that of the cooling medium.


\subsection{Types of germanium detectors}
\label{sec:det:type}
Germanium detectors fall naturally into two main classes characterized by the active impurities in the bulk material after the crystal pulling process. If the impurities are mainly acceptors, the detector is called $p$-type. If the impurities are mainly donors, the detector is called $n$-type. 

Large germanium detectors normally have a cylindrical shape. They can be divided into two types: \textit{true coaxial} and \textit{closed-ended coaxial}. In both cases the shape is cylindrical with an inner bore.\footnote{There are also some germanium detectors having no hole in the middle. All the contacts are on the surface.} For true coaxial detectors the core is completely removed, whereas for the closed-ended coaxial geometry the core is only partially removed leaving a \textit{cap} on one side.

The outer surface layer of a $p$-type detector is normally converted into $n$-type with a typical thickness of about 0.5~mm by diffusing lithium in. It is sometimes called the \textit{dead layer}, because the charge carriers created in this layer cannot be detected. The inner surface of a $p$-type detector is implanted with boron in order to make good electric contact. The thickness of the implantation is of the order of several microns. $n$-type detectors have the dead layer on the inner surface, hence have less inactive volume. The boron implantation is on the outside.

Germanium detectors can also be classified as segmented or unsegmented. The segmentation is normally performed on the outer surface of the detector. For $p$-type detectors the outer surface is milled in order to penetrate the thick lithium-drifted layer. The fringe depths and thicknesses of the segments are of the order of a millimeter. Distortions in the electric field are expected in this case. For $n$-type detectors photo-lithographic techniques are used to form the segment boundaries on the outer surface implantation. The width of the boundary is of the order of several microns.

\section{Drift of charge carriers}
\label{sec:det:drift}
\subsection{Creation of charge carriers}
\label{sec:det:exit}
A big fraction of the energy deposited by incident radiations causes the excitation of phonons, the rest excites electrons from the valence band to the conduction band. Thus, the energy needed to create one electron-hole pair, called pair energy, is much bigger than the band gap. In germanium the pair energy is 2.95~eV at 80~K which is about four times larger than the band gap.

\subsection{Electric field}
\label{sec:det:field}
The reverse bias applied to the electrodes creates an electric field in the bulk of the germanium detector which lets the charge carriers drift to the electrodes. The field $\mathbf{E}$ is determined by the boundary conditions as well as the space charges in the depletion zone. It can be calculated by solving Poisson's equation:
\begin{equation} 
  \label{eq:det:ef}
  \nabla \cdot \mathbf{E} = \frac{\rho}{\epsilon},  
\end{equation}
where $\rho$ is the space charge density defined by the effective number of impurities and $\epsilon$ is the dielectric constant.

\subsection{Effects of crystal structure}
\label{sec:det:struc}
The drift velocities of charge carriers, $\mathbf{v}_{e/h}(\mathbf{r})$, are related to the electric field, $\mathbf{E}(\mathbf{r})$, by a coefficient named \emph{mobility}, $\mu_{e/h}$:
\begin{equation} 
  \label{eq:det:dv}
  \mathbf{v}_{e/h} (\mathbf{r})= \mu_{e/h} \mathbf{E}(\mathbf{r}),
\end{equation}
where $\mathbf{r}$ indicates the position. In germanium detectors operating at low temperature the mobility is influenced by the crystal lattice orientation. The drift velocities along different directions differ from each other (longitudinal anisotropy) and are not always parallel to the electrical field (transversal anisotropy). The angle between the drift direction and the electric field is known as the Sasaki angle\cite{Sas56}. A detailed discussion of the effects of the crystal structure on the charge carrier drift is contained in Chapter~\ref{cha:pss}.

\section{Induction of signals in detector electrodes}
\label{sec:det:ramo}
Electric signals are induced in the electrodes of a detector by the cumulative influence of electrons and holes moving toward the electrodes. Shockley-Ramo's Theorem~\cite{Gat82, Rad88, He00} can be used to calculate the time development of the induced charge $Q(t)$ or current $I(t)$ in each electrode:
\begin{equation} 
  \label{eq:det:ramoq}
  Q(t) = -Q_{0} \times [\varphi_{w}(\mathbf{r}_{h}(t)) - \varphi_{w}(\mathbf{r}_{e}(t))],
\end{equation}
\begin{equation} 
  \label{eq:det:ramoi}
  I(t) = Q_{0} \times [\mathbf{E}_{w}(\mathbf{r}_{h}(t)) \cdot \mathbf{v}_{h}(t) - \mathbf{E}_{w}(\mathbf{r}_{e}(t)) \cdot \mathbf{v}_{e}(t)],
\end{equation}
where $Q_{0}$ is the electric charge carried by electrons or holes, $\mathbf{r}_{e/h}(t)$ and $\mathbf{v}_{e/h}(t)$ are the position and velocity vectors of electrons/holes as a function of time, $\varphi_{w}(\mathbf{r})$ and $\mathbf{E}_{w}(\mathbf{r})$ are the so-called \emph{weighting potentials} and \emph{weighting fields}. They can be calculated by solving Poisson's equations, $\nabla^{2} \varphi(\mathbf{r}) = 0$ and $\nabla \cdot \mathbf{E}(\mathbf{r}) = 0$, with the boundary condition that the potential on the electrode of interest equals to 1 and the potential on any other electrode equals to 0.

\section{Effects of electronics on signal formation}
\label{sec:det:elec}

\subsection{Noise}
\label{sec:det:noise}
The total energy resolution in terms of the full width at half maximum
(FWHM) of the peak under study, $W_{T}$, is composed of three terms:
\begin{equation}
W_{T}^{2} = W_{D}^{2} + W_{X}^{2} + W_{E}^{2},
\end{equation}
where $W_{D}$ describes the statistical fluctuations of the creation of electron-hole pairs, $W_{X}$ describes the effect of incomplete charge collection and scales linearly with the incident energy, $W_{E}$ accounts for noise contributions from the electronics and is constant in energy. Given a perfect electronic system, germanium detectors have an energy resolution of about 2~keV at 1.3~MeV, where $W_{X}$ dominates the contribution and $W_{D}$ contributes less than 1~keV. These two contributions cannot be reduced. In practise, thermal noise in pre-amplifiers and noise picked up by the cables and connecters necessary for the signal transmission contribute significantly to the total noise. These contributions have to be kept as small as possible.

\subsection{Cross talk}
\label{sec:det:xtalk}
For segmented germanium detectors, signals from all the electrodes are read out simultaneously. There are intrinsic cross talks between different channels because of the electric couplings between electrodes. These cross talks are not due to improper electric connections, hence cannot be avoided. However, they can be reduced to an acceptable level by choosing proper values of resistors and capacitors used in the pre-amplifier coupling or feed-back circuits to match the intrinsic capacities between detector electrodes. A detailed analysis in found in Chapter 4 in Ref.~\cite{Bru06} and references therein.


%%% Local Variables:
%%% mode:latex
%%% TeX-master: "thesis"
%%% End:
