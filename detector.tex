When particles, $\alpha, \beta, \gamma, n, p$, etc., interact with the germanium semiconductor detector, they will create electron-hole pairs, which are also called charge carriers. Driven by the external electric field applied to the detector, the charge carriers will drift, and induce electric signals in electrodes. The electric signals will be amplified, digitized and recorded by the electronics and data acquisition system connected to the germanium detector, and ready for the analysis. The whole process of the signal formation in the germanium detector and its surrounding electronics will be briefly reviewed in this chapter.

\section{Interactions of radiation with matter}
\label{sec:det:phys}

\section{Germanium semiconductor}
\label{sec:det:semi}

\section{Drift of charge carriers}
\label{sec:det:drift}

\section{Induction of signals in detector electrodes}
\label{sec:det:lamo}

\section{Effects of electronics on signal formation}
\label{sec:det:elec}

\section{Summary}
\label{sec:det:sum}



%%% Local Variables:
%%% mode:latex
%%% TeX-master: "thesis"
%%% End:
