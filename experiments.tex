\chapter{Neutrinoless double beta decay experiments}
\label{cha:exps}
Neutrinoless double beta decay is an extremely rare process even if it
does exist. Requirements, which are crucial to increase the
sensitivity of $0\nu\beta\beta$ decay experiments, are summarized. The
``Pros and Cons'' of different experimental approaches are discussed
based on these requirements.

\section{Sensitivity}
\label{sec:sensi}
The number of observed $0\nu\beta\beta$ decay events $N_{s}$ within
the measuring time $t$, can be calculated as
\begin{equation}
\label{eq:gerda:ns}
N_{s} = M \cdot \kappa \cdot \frac{N_{A}}{M_{A}} \cdot \epsilon 
\cdot (1 - e^{-t/\tau}) \approx M \cdot \kappa \cdot 
\frac{N_{A}}{M_{A}} \cdot \epsilon \cdot \frac{t}{\tau},
\end{equation}
where, $M$ is the total mass of the source material, $\kappa$ is the
mass fraction of the isotope under study, $N_{A}$ is Advogadro's
number, $M_{A}$ is the atomic mass of the isotope, $\epsilon$ is the
signal detecting efficiency, and $\tau$ is the mean lifetime of the
decay. Since the measuring time $t$ is much shorter than the mean
lifetime $\tau$, $(1 - e^{-t/\tau})$ is approximated as $t/\tau$. The
half lifetime, $T^{0\nu}_{1/2}$, is then
\begin{equation}
\label{eq:gerda:thalf}
T^{0\nu}_{1/2} = \ln2 \cdot \tau \approx \ln2 \cdot M \cdot \kappa 
\cdot \frac{N_{A}}{M_{A}} \cdot \epsilon \cdot \frac{t}{N_{s}}.
\end{equation}
The number of background events within the measuring time $t$ and
within the energy window of interest $\Delta E$ is
\begin{equation}
  \label{eq:gerda:nb}
  N_{b} = b \cdot M \cdot t \cdot \Delta E,
\end{equation}
where $b$ is the background index given in per kilogram of source
material per measuring year and per keV. If $N_{s}$ is smaller than
the standard fluctuation expected for $N_{b}$, \textit{i.e.},
$N_{s}<\sqrt{N_{b}}$\footnote{The standard fluctuation expected for
$N_{b}$ can be expressed as $\sqrt{N_{b}}$ only if $N_{b} \gtrsim
10$. A detailed study of the sensitivity with small $N_{b}$ can be
found in Ref.~\cite{Cal06}}, the signal cannot be extracted. In this
case the relation
\begin{equation}
\label{eq:gerda:thalfb}
T^{0\nu}_{1/2} > \ln2 \cdot M \cdot \kappa \cdot \frac{N_{A}}{M_{A}} 
\cdot \epsilon \cdot \frac{t}{\sqrt{N_{b}}} = \ln2 \cdot \kappa \cdot 
\frac{N_{A}}{M_{A}} \cdot \epsilon \sqrt{\frac{M t}{b \Delta E}}
\end{equation}
can be used to set a lower limit on the half lifetime. Combined with
Eq.~\ref{eq:0nurate}, the following relation can be deduced to set an
upper limit on the effective Majorana neutrino mass:
\begin{equation}
\label{eq:gerda:mbb}
m_{\beta\beta} < \sqrt{\frac{M_{A}}{\ln2 \cdot \kappa \cdot N_{A} \cdot 
\epsilon}} \sqrt{\frac{1}{G_{0\nu}(Q,Z)}} \frac{1}{|\mathcal{M}_{0\nu}|} 
(\frac{b \Delta E}{M t})^{1/4}
\end{equation}
The sensitivity of a $0\nu\beta\beta$ decay experiment in terms of
lifetime or neutrino mass can be estimated based on
Eq.~\ref{eq:gerda:thalfb} or \ref{eq:gerda:mbb}, respectively.

\section{Experimental approaches}
\label{sec:exp:appr}
\subsection{General considerations}
\label{sec:gencon}
An analysis of Eq.~\ref{eq:gerda:mbb} provides guidance on how to
design a $0\nu\beta\beta$ decay experiment with a good sensitivity. As
many of the following requirements should be met:
\begin{itemize}
\item the mass of the source material $M$ should be large;
\item the abundance of the isotope under study should be high, either
naturally or by enrichment;
\item the calculation of the nuclear matrix element
$|\mathcal{M}_{0\nu}|$ for this isotope should be accurate;
\item the $Q$-value should be large, because $G_{0\nu}(Q,Z) \propto
Q^{5}$, and the higher the $Q$-value, the fewer lines from natural
radioactivity produce background;
\item the signal detecting efficiency should be large;
\item the energy resolution should be good in order to allow a small
$\Delta E$;
\item last but not the least, the background level $b$ should be as
low as possible.
\end{itemize}
And then patience is needed; data is generally collected over many
years.

Except for background suppression techniques, the experimental
approaches are mainly determined by the choice of the source
material. Table~\ref{tab:gerda:iso} presents a selection of isotopes
used or planned to be used to search for $0\nu\beta\beta$ decay. Also
listed are their $Q$-values, nuclear matrix elements \cite{Mut90,
Rod07, Sim08, Cau08}, natural abundance $\kappa_{0}$ and properties
important for the experimental design. Different experimental
approaches can be classified into two categories: 1. the source
material can be used to produce the detector; 2. the source is not the
detector, the decay products need to be detected using equipment
around the source.
\begin{table}[htbp]
\centering
\caption{A selection of possible source candidates for
$0\nu\beta\beta$ decay experiments. Also listed are their $Q$-values,
nuclear matrix elements, natural abundance $\kappa_{0}$ and properties
important for the design of experiments.}
\label{tab:gerda:iso}
\begin{minipage}{\linewidth}
\begin{tabular}{ccccc} \hline Isotope & $Q$ [MeV] &
$\mathcal{M}_{0\nu}$ & $\kappa_{0}$ [\%] & Properties \\\hline
$^{48}$Ca & 4.271 & 0.67\footnote{The values are from an ISM 
(Interacting Shell Model) calculation in Ref~\cite{Cau08}. 
The other $\mathcal{M}_{0\nu}$ values with errors are from QRPA 
(Quasi-particle Random Phase Approximation) calculations 
\cite{Rod07}. The errors are from the measurement of 
$2\nu\beta\beta$ experiments.} & 0.19 & CaF$_{2}$ \& 
CaWO$_{4}$ is a scintillator \\
$^{76}$Ge & 2.039 & $4.51 \pm 0.17$ & 7.8 & semiconductor \\
$^{82}$Se & 2.995 & $4.02 \pm 0.15$ & 9.2 & - \\
$^{96}$Zr & 3.350 & $1.12 \pm 0.03$ & 2.8 & - \\
$^{100}$Mo & 3.034 & $3.34 \pm 0.19$ & 9.6 & - \\
$^{116}$Cd & 2.809 & $2.74 \pm 0.19$ & 7.5 & 
CdZnTe\footnote{There are other isotopes in the CdZnTe crystal
that could undergo $0\nu\beta\beta$ decay. The rest of them 
are $^{70}$Zn with $Q = 1.001$~MeV, $\kappa_{0} = 0.62\%$,
$^{114}$Cd with $Q = 0.534$~MeV, $\kappa_{0} = 28.7\%$, $^{128}$Te
with $Q = 0.868$~MeV, $\kappa_{0} = 31.7\%$ and $^{130}$Te.} is a 
semiconductor;\\
& & & & CdWO$_{4}$ is a scintillator\\
$^{124}$Sn & 2.287 & $2.11^{a}$ & 5.8 & semiconductor \\
$^{130}$Te & 2.530 & $3.26 \pm 0.12$ & 35 & TeO$_{2}$ can be 
used as bolometer\\
$^{136}$Xe & 2.480 & $2.11 \pm 0.11$ & 8.9 & active material for 
time projection chambers\\
$^{150}$Nd & 3.367 & $4.74 \pm 0.20$ & 5.6 & could be dissolved 
in liquid scintillator\\
\end{tabular}
\end{minipage}
\end{table}

\subsection{Source and detector are identical}
\label{sec:exp:sed}
As shown in Table~\ref{tab:gerda:iso}, quite a few $0\nu\beta\beta$
decay candidates have special properties which allow them to be used
as detectors. There are advantages to this concept. As the decay
electrons do not have to leave the source and reach the detector, the
detection efficiency is not limited and the energy resolution of the
detector not deteriorated. As a consequence large compact masses are
useable limiting the loss of events close to a surface with electrons
escaping. The drawback is that such detectors usually have limited
capability to reconstruct event topologies and normally only one
isotope can be studied.

$^{48}$Ca has the highest $Q$-value among all the candidates. Hence
low background from natural radioactivities is expected. It also means
a large phase space factor which enlarges the $0\nu\beta\beta$ decay
rate for a given Majorana mass. However, until now only few
experiments have been carried out because of its low natural
abundance. The most stringent limit on the $0\nu\beta\beta$ decay of
$^{48}$Ca came from ELEGANT VI \cite{Oga04} using CaF$_{2}$
scintillator. Two future experiments using CaF$_{2}$ and CaWO$_{4}$ as
scintillator, respectively, are CANDLES \cite{Hir08} and
CARVEL \cite{Zde05}. They aim at a sensitivity in $m_{\beta\beta}
<$~(0.04-0.09)~eV.

The search for the $0\nu\beta\beta$ decay of $^{76}$Ge is affected by
natural radioactivity due to its low $Q$-value. Enrichment in
$^{76}$Ge is also needed in order to overcome the low natural
abundance. However, semiconductor detectors made from high purity
germanium crystals have been used as gamma spectrometers for years and
have an excellent energy resolution. Previous $^{76}$Ge
$0\nu\beta\beta$ decay experiments include IGEX \cite{Aal02} and
HdM \cite{Hei04}. GERDA \cite{Sch05} Phase I is currently under
construction and will be described in detail in the next chapter. The
planned future experiments include GERDA Phase II and
Majorana \cite{Gai03, Aal04}. The GERDA and Majorana Collaborations
have reached an agreement to share resources and knowledge where
appropriate in their parallel development of the two different
detector designs. The ultimate goal is to combine the strength of the
two collaborations in a future experiment that will employ the best
technology for reaching a Majorana neutrino mass sensitivity of below
0.05~eV.

The Cobra experiment \cite{Zub01, Ell02, Kie03} is a special case in
the ``source = detector'' concept. A large array of CdZnTe
semiconductor detectors is going to be used. The CdZnTe crystal
contains 5 isotopes which could undergo $0\nu\beta\beta$
decay. Pixellated CdZnTe detectors can be operated as solid-state time
projection chambers (TPC) and hence offer tracking capability which
allows reconstruction of the event topology. Another advantage of
CdZnTe detectors is that they can be operated at room temperature. No
large and complicated cooling facility is needed. However, the energy
resolution of CdZnTe detectors currently available is not as good as
those of germanium detector and TeO$_{2}$ bolometers.

The tungstate CdWO$_{4}$ is similar to CaWO$_{4}$, and can be used as
scintillator. A comprehensive comparison between them can be found on
page 27 of Ref.~\cite{Avi05}. The CdWO$_{4}$ crystal is less
contaminated and has better background/signal discrimination power
than the CaWO$_{4}$ crystal. Previous $0\nu\beta\beta$ decay
experiments using CdWO$_{4}$ scintillators include the one performed
by the Kiev-Florence collaboration in the Solotvina Underground
Laboratory since 1989 \cite{Dane00, Dan03} and CAMEO \cite{Bel00,
Bel01}. The CAMEO project also proposes to exploit 1 ton of
$^{116}$CdWO$_{4}$ detectors placed in one of the large underground
neutrino detectors such as BOREXINO \cite{Arp08}, SNO or KamLAND. The
sensitivity is estimated to be $m_{\beta\beta} < 0.02$~eV.

CUORICINO \cite{Pre04}, the pilot experiment for CUORE \cite{Arn04,
Ard05}, just released a new upper limit of $m_{\beta\beta}$
\cite{Arn08} using TeO$_{2}$ bolometers. TeO$_{2}$ bolometers have
almost the same energy resolution as the germanium
detector. $^{130}$Te has the highest natural abundance among all the
$0\nu\beta\beta$ decay candidates. CUORE is also trying to use
CdWO$_{4}$ as scintillating bolometer \cite{Gir08}. The scintillating
bolometer could provide more parameters helping background rejection,
especially from surface contamination. One of the challenges of CUORE
is to stabilize the contact between the bolometer and the thermometer.

Both gaseous and liquid xenon can be used as active material in a TPC,
the later can also be used as a scintillator. The XMASS experiment
\cite{Kim05}, currently under construction in the Kamioka Observatory,
is going to use liquid xenon as a scintillator. The position of a
interaction can be extracted from the hit pattern of the PMTs around
the scintillator. The EXO experiment \cite{Dani00}, also under
construction currently, is going to use TPC filled with liquid xenon
which can be used as scintillator at the same time. The position of a
interaction can be inferred from the time difference of the
scintillation light signal and the ionization signal. In gaseous xenon
it is possible to reconstruct the tracks of the electrons from the
beta decay. This makes it easy to discriminate against background
induced by natural radioactivity. An experiment was carried out in the
Gotthard underground laboratory using TPC filled with xenon gas
enriched to 62.5~\% in $^{136}$Xe at a pressure of 5 bar \cite{Lue98}.

$^{150}$Nd has the second highest $Q$-value, the largest nuclear
matrix element $\mathcal{M}_{0\nu}$ (though the uncertainty is large)
and a relatively high natural abundance (enrichment is possible) among
all the $0\nu\beta\beta$ decay candidates. In a follow-up experiment
to SNO, called SNO+ \cite{Zub07}, the old SNO infrastructure will be
filled with Nd-loaded liquid scintillator instead of
D$_{2}$O. Although the energy resolution of the detector will not be
as good as that of other existing experiments, the mass that could be
suspended in the scintillator is large (0.1\% load of enriched Nd
corresponds to about 500~kg of $^{150}$Nd). This may give SNO+ the
sensitivity of $m_{\beta\beta}$ as low as 0.03~eV.

\subsection{Source and  detector are not identical}
\label{sec:exp:sued}
The advantages of using external detectors are (1) the source material
could be changed so that several $0\nu\beta\beta$ decay candidates
could be investigated in the same experiment; (2) using tracking
devices the event topology could be reconstructed which leads to
excellent background discrimination. The disadvantages are (1)
normally the energy resolution is not good; (2) in order to release
the beta decay electrons, the source has to be made into thin foils
and hence large masses are difficult to integrate in an experiment.

The previous experiments include TGV I and II \cite{Ste98, Ste00,
Ste06}, NEMO I \cite{Das91}, II \cite{Arn95} and III \cite{Arn05,
Arn07}. In TGV Cd plates were put in between germanium spectrometers.
In NEMO the source foils (Ca, Se, Zr, Cd, Mo, Te and Nd) were fixed
between two tracking volumes composed of many drift cells. The planned
future experiments include SuperNEMO \cite{Sne08}, MOON \cite{Nak06}
and DCBA \cite{Ish05}. Based on the NEMO III experience, SuperNEMO
aims at the sensitivity of $m_{\beta\beta} \sim 0.03$~eV. In MOON,
enriched $^{100}$Mo foils will be interleaved with plastic
scintillators which work as a calorimeter as well as an active shield.
The designed sensitivity of MOON is $m_{\beta\beta} \sim 0.03$~eV. In
DCBA thin source plates ($^{150}$Nd,$^{100}$Mo $^{82}$Se) will be
installed in tracking chambers located in a uniform magnetic
field. DCBA is an R\&D project for MTD (Magnetic Tracking Detector),
the design sensitivity of which is $m_{\beta\beta} \sim
0.02$-0.07~eV \cite{Ish07}.

\subsection{Summary}
\label{sec:exp:comp}
Table~\ref{tab:gerda:comp} summarizes the proposed $0\nu\beta\beta$
decay experiments mentioned in the previous discussion. The schedule
is not clear in most cases. However, the table provides a quick
reference concerning the experimental aspects of $0\nu\beta\beta$
decay research.
  
\begin{table}[htbp]
\centering
\caption{Characteristics of proposed $0\nu\beta\beta$ experiments.
The corresponding references are: CARVEL \cite{Zde05}, CANDLES
\cite{Hir08}, GERDA \cite{Sch05,Cal06}, Majorana \cite{Aal04}, Cobra
\cite{Kie03}, CAMEO \cite{Bel01}, CUORE \cite{Ard05}, XMASS
\cite{Nak02}, EXO \cite{Dani00}, SNO+ \cite{Zub07}, MOON \cite{Nak06},
SuperNEMO \cite{Sne08}, DCBA/MTD \cite{Ish07}}
\label{tab:gerda:comp}
\begin{minipage}{\linewidth}
\begin{tabular}{lllc} \hline 
Experiment & Technique & Sensitivity\footnote{Please refer to the 
references for the definition of the sensitivities for the individual 
experiment.} ($m_{\beta\beta}$/eV) & Schedule \\\hline
CARVEL & CaWO$_{4}$  scintillator & 0.04-0.09 & - \\
CANDLES & CaF$_{2}$ scintillator & - & - \\
GERDA Phase I & Ge detector in LAr\footnote{liquid argon} & 0.27 & 2009 \\
GERDA Phase II & Ge detector in LAr & 0.11 & - \\
Majorana & $^{76}$Ge detector & 0.03-0.04 & - \\
Cobra & CdZnTe semiconductor & $< 1$ & - \\
CAMEO & CdWO$_{4}$ scintillator & $< 0.02$ & - \\
CUORE & TeO$_{2}$ bolometer & $< 0.03$ & - \\
XMASS & liquid Xe TPC & 0.06-0.09 & - \\
EXO & liquid Xe TPC with laser tagging & $< 0.01$ & - \\
SNO+ & Nd-load scintillator & $< 0.05$ & 2010 \\ 
MOON & Mo foil interleaved with scintillators & $\sim0.03$ & -\\
SuperNEMO & drift chamber + calorimeter & $< 0.05$ & 2012 \\
DCBA/MTD & source plates in drift chamber & 0.02-0.07 & 2016 \\
\end{tabular}
\end{minipage}
\end{table}

%%% Local Variables:
%%% mode:latex
%%% TeX-master: "thesis"
%%% End:
