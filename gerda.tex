% $Id$
\section{Neutrinoless double beta decay experiments}
\label{sec:gerda:nonubb}

\subsection{Sensitivity}
\label{sec:gerda:sensi}
The number of observed neutrinoless double beta decay events as a function of the measuring time, $N(t)$, can be calculated as
\begin{equation}
  \label{eq:gerda:nt}
  N(t) = M \cdot \kappa \cdot \frac{N_{A}}{M_{A}} \cdot \epsilon \cdot (1 - e^{-t/\tau}) \approx M \cdot \kappa \cdot \frac{N_{A}}{M_{A}} \cdot \epsilon \cdot \frac{t}{\tau},
\end{equation}
where, $M$ is the total mass of the source material, $\kappa$ is the mass fraction of the isotope under study, $N_{A}$ is Advogadro's number, $M_{A}$ is the atomic mass of the isotope, $\epsilon$ is the signal detecting efficiency, and $\tau$ is the mean lifetime of the decay. Since the measuring time $t$ is much shorter than the mean lifetime $\tau$, $(1 - e^{-t/\tau})$ is approximately to be $t/\tau$. The half lifetime, $T^{0\nu}_{1/2}$, is then
\begin{equation}
  \label{eq:gerda:hlt}
  T^{0\nu}_{1/2} = \ln2 \cdot \tau \approx \ln2 \cdot M \cdot \kappa \cdot \frac{N_{A}}{M_{A}} \cdot \epsilon \cdot \frac{t}{N(t)},
\end{equation}


\subsection{General consideration}
\label{sec:gerda:gencon}

\subsection{Source is detector}
\label{sec:gerda:sed}

\subsection{Source is not  detector}
\label{sec:gerda:sued}

\section{The GERDA experiment}
\label{sec:gerda:concept}

\subsection{Concept}
\label{sec:gerda:concept}

\subsection{Status}
\label{sec:gerda:status}


%%% Local Variables:
%%% mode:latex
%%% TeX-master: "thesis"
%%% End:
