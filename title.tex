%$Id$

\begin{titlepage}

\centering

\vspace{1.0cm}

\begin{center}

{\Huge Technische Universit\"at M\"unchen}

\vspace{1.5 cm}

{\Large Max-Planck-Institut f\"ur Physik \\}

\vspace{0.5 cm} 

{\Large (Werner-Heisenberg-Institut)} 

\vspace{1.5 cm}

{\Large Techniques to distinguish between electron and photon induced
events using segmented germanium detectors}

\vspace{1.5 cm}

{\Large Jing Liu\\}

\end{center} 

\vspace{1.5 cm}

\begin{center}
Vollst\"andiger Abdruck der von der Fakult\"at f\"ur Physik der
Technischen Universit\"at M\"unchen zur Erlangung des akademischen
Grades eines \\Doktors der Naturwissenschaften (Dr. rer. nat.) \\ 
genehmigten Dissertation. \\
\end{center}

\vspace{1.5 cm} 

\begin{table*}[h]
\center
\begin{tabular}{ll}
Vorsitzender: & \phantom{1.} Univ.-Prof. Michael Ratz\\ 
& \\ 
Pr\"ufer der Dissertation: & \\ 
& 1. Hon.-Prof. Allen C. Caldwell, Ph.D \\ 
& 2. Univ.-Prof. Lothar Oberauer \\ 
\end{tabular}
\end{table*}

\vspace{2.0 cm} 

\begin{center}
Die Dissertation wurde am 26.04.2007 bei der Technischen Universit\"at
M\"unchen eingereicht und durch die Fakult\"at f\"ur Physik am
05.06.2007 angenommen. \\
\end{center}

\end{titlepage} 

%%% Local Variables:
%%% mode:latex
%%% TeX-master: "thesis"
%%% End:
