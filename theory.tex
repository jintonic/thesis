%$Id$
Neutrinos were introduced into the \emph{Standard Model} as massless
particles. But strong evidences from neutrino oscillations expriments
showed that at least one of the neutrinos had mass. In order to
include neutrino mass into the \emph{Standard Model} without
conflicting with the fact that neither right handed neutrinos nor left
handed antineutrinos are observed, neutrinos are assume to be of
Majorana type, \textit{i.e.} their own anti-particles. Neutrinoless
double beta decay would exist only if this assumption is true. And
from the rate of Neutrinoless double beta decay the effective Majorana
neutrino mass can be deduced. This can help to solve the neutrino mass
hierarchy problem that cannot be solved by the neutrino oscillation
experiment alone. There are approaches other than neutrinoless double
beta decay which can also address on the same problem. But they are
not quite practical for the time being.

\section{Neutrinos in \emph{Standard Model}}
\label{sec:sm}
In the \emph{Standard Model} neutrinos are assumed to be fermions with
spin 1/2 and rest mass $m_\mu=0$. They always have fixed helicities
because there is no frame of reference moving faster than a neutrino,
where the helicity of the neutrino could change its sign. Neutrinos
and anti-neutrinos are believed to be different particles. Lepton
numbers +1 and -1 are asssigned to them respectively and the sum of
which is required to be conservative in order to indicate the
difference. Only left-handed neutrinos and right-handed anti-neutrinos
participate the weak interaction. The field operators of right-handed
neutrinos and left-handed anti-neutrinos do not present in the
Lagrangian density of weak interaction at all.

New experimental evidences, particularly neutrino oscillations, make
the modification and extension of the \emph{Standard Model} necessary.
Crucial theoretical considerations and experimental observations of
neutrinos in history leading to the above statements are briefly
reviewed below in order to investigate the space allowed for new
ideas.

The neutrino was postulated to exist by W. Pauli in 1930 in order to
explain the continuous energy spectrum of electrons emitted from the
beta decay without abandoning the law of energy conservation. He
assumed that it was a neutral fermion with spin 1/2, and its mass was
of the same order of magnitude as the electron mass.~\cite{Pau30} E.
Fermi soon developed his theory of beta decay according to the beta
spectrum~\cite{Fer33,Fer34}. He investigated the influence of the
neutrino mass on the shape of beta spectrum and inferred that $m_\nu
\approx 0$ by comparing the calculation with the experimental data. A
precise measurement of the beta spectrum of tritium by L. Langer and
R. Moffat in 1952~\cite{Lan52} gave an upper limit on the rest mass of
the neutrino, $m_\nu < 250 \mbox{eV} = 0.002m_e$. The neutrino was
assumed to be massless afterwards. Although the upper limit was pushed
down again and again by the later experiments, the possibility that
neutrinos have very tiny masses have never been completely ruled out.

The obsorption of free neutrinos was observed in 1956 by C. L. Cowan,
Jr. and F. Reines.~\cite{Rei58}. Using enormous amount of
anti-neutrinos from nuclear reactors they overcome the difficulties
arising from the small reaction cross section and observed the
reaction,
\begin{equation}
  \label{eq:nup2en}
  \bar{\nu_e}+p \rightarrow e^{+}+n.
\end{equation}


For massless fermions there was a so-called two-component model, which
was mentioned for the first time by H. Weyl~\cite{Wey29}. According to
this model only the two right-handed components or the two left-handed
components exist in nature. It was always rejected because it
completely violated the parity conservation in the interaction, which,
at that time, was believed to be true. This model was applied to the
neutrino again by Lee and Yang~\cite{Lee57}, and some other
authors~\cite{Sal57,Lan57} soon after the parity
violation~\cite{Lee56} was observed in the beta decay of
$^{60}$Co~\cite{Wu57} and the creation and decay of
muons~\cite{Gar57,Fri57}. This idea was widely accepted after an
elegant experiment carried out by M. Goldhaber \textit{et
  al.}~\cite{Gol58}. By measuring the polarization of $\gamma$-rays
emitted from $^{152}$Sm* created in the electron capture
$^{152}$Eu$(e^-,\nu)$ they inferred that neutrinos from
$^{152}$Eu$(e^-,\nu)$ were left-handed.

Based on these theoretical development and experimental observations
the \emph{Standard Model} of weak interaction was established. In this
model neutrinos are massless and only left-handed neutrinos and
right-handed anti-neutrinos participate the weak interaction.

\section{neutrino oscillations}
\label{sec:osci}

\section{Majorana neutrinos}
\label{sec:major}

\section{Neutrino mass terms}
\label{sec:mass}

\section{Neutrinoless double beta decay}
\label{sec:0n2b}

\subsection{Current state}
\label{sec:state}

\section{Other approaches to probe neutrino types}
\label{sec:others}



%%% Local Variables:
%%% mode:latex
%%% TeX-master: "thesis"
%%% End:
