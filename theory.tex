%$Id$
Neutrinos were introduced into the \emph{Standard Model} as massless
particles. But strong evidences from neutrino oscillations expriments
showed that at least one of the neutrinos had mass. In order to
include neutrino mass into the \emph{Standard Model} without
conflicting with the fact that neither right handed neutrinos nor left
handed antineutrinos are observed, neutrinos are assume to be of
Majorana type, \textit{i.e.} their own anti-particles. Neutrinoless
double beta decay would exist only if this assumption is true. And
from the rate of Neutrinoless double beta decay the effective Majorana
neutrino mass can be deduced. This can help to solve the neutrino mass
hierarchy problem that cannot be solved by the neutrino oscillation
experiment alone. There are approaches other than neutrinoless double
beta decay which can also address on the same problem. But they are
not quite practical for the time being.

\section{Neutrinos in \emph{Standard Model}}
\label{sec:sm}
In order to explain the continuous energy spectrum of electrons
emitted from the $\beta$-decay without abandoning the law of energy
conservation W. Pauli postulated in 1930 that there was another
particle except for the eletron emitted in the $\beta$-decay. He
called it ``neutron'' and assumed that it should have spin 1/2, its
mass be of the same order of magnitude as the electron mass, and its
penetrating power be 1-10 times larger than a $\gamma$-ray.
\cite{Pau30} E. Fermi renamed it ``neutrino'' to distinguish it from
the heavier neutron discovered by J. Chadwick in 1932 and developed
his theory of $\beta$-decay according to the $\beta$-spectrum
\cite{Fer33,Fer34}. He investigated the influence of the neutrino mass
on the shape of $\beta$-spectrum. F. Perrin carried out similar
results at the same time independently \cite{Per33}. They inferred
that $m_\nu \approx 0$ by comparing their calculation with the
experimental $\beta$-spectrum. A precise measurement of the
$\beta$-spectrum of tritium by L. Langer and R. Moffat in 1952
\cite{Lan52} gave an upper limit on the rest mass of the neutrino,
$$m_\nu < 250 \mbox{eV} = 0.002m_e.$$
The neutrino was assumed to be massless afterwards.


\section{neutrino oscillations}
\label{sec:osci}

\section{Majorana neutrinos}
\label{sec:major}

\section{Neutrino mass terms}
\label{sec:mass}

\section{Neutrinoless double beta decay}
\label{sec:0n2b}

\subsection{Current state}
\label{sec:state}

\section{Other approaches to probe neutrino types}
\label{sec:others}



%%% Local Variables:
%%% mode:latex
%%% TeX-master: "thesis"
%%% End:
